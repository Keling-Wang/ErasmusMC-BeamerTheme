\documentclass{beamer}
\title{\textbf{Erasmus MC Beamer Theme}}
\subtitle{modified from Tsinghua Beamer theme}


\usepackage{hyperref}
\usepackage[T1]{fontenc}

% other packages
\usepackage{latexsym,amsmath,amssymb,amsfonts,multicol,booktabs,calligra}
\usepackage{graphicx,pstricks,listings,stackengine}
\usepackage[backend=biber,style=nejm]{biblatex}
\addbibresource{ref.bib}
\renewcommand{\cite}{\parencite}

\usepackage{erasmusmc}
%%% Author, Affiliation, Date
\author{Jane Doe\inst{1}}
\institute{\inst{1} Department of Epidemiology, Erasmus MC Rotterdam}
\date{30 May 2024}



%%% defs
\def\cmd#1{\texttt{\color{red}\footnotesize $\backslash$#1}}
\def\env#1{\texttt{\color{blue}\footnotesize #1}}
%\definecolor{erasmusmc}{RGB}{134,210,236}
%\definecolor{erasmusmc_dark}{RGB}{12,33,116}
\definecolor{deepblue}{rgb}{0,0,0.5}
\definecolor{deepred}{rgb}{0.6,0,0}
\definecolor{deepgreen}{rgb}{0,0.5,0}
\definecolor{halfgray}{gray}{0.55}

\lstset{
    basicstyle=\ttfamily\small,
    keywordstyle=\bfseries\color{erasmusmc_dark},
    emphstyle=\ttfamily\color{erasmusmc!50!erasmusmc_dark},    % Custom highlighting style
    stringstyle=\color{erasmusmc!50!deepgreen},
    numbers=left,
    numberstyle=\small\color{halfgray},
    rulesepcolor=\color{red!20!green!20!blue!20},
    frame=shadowbox,
}

\renewcommand{\familydefault}{\sfdefault} % You can define your font style here




\begin{document}

\begin{frame}
    %\sffamily
    \titlepage
    \vspace{-1.5em}
    \begin{figure}[htpb]
        \begin{center}
            \includegraphics[width=0.35\linewidth]{pic/erasmusmc_logo.pdf}
        \end{center}
    \end{figure}
\end{frame}

\logo{\includegraphics[width=.2\textwidth]{pic/erasmusmc_logo.pdf}}

\begin{frame}
    \tableofcontents[sectionstyle=show,subsectionstyle=show/shaded/hide,subsubsectionstyle=show/shaded/hide]
    
\end{frame}



\section{Background}

\begin{frame}{\textbf{Feeling cool when using Beamer?}}
    \begin{itemize}[<+-| alert@+>] % other than alert it's also okay to insert \pause manually.
        \item We all know \LaTeX{}, and many institutes have their own themes.
        \item For Chinese compilation support, use Xe\LaTeX{}
        \item The original THU Beamer Overleaf project: \url{https://www.overleaf.com/latex/templates/thu-beamer-theme/vwnqmzndvwyb}, ready to use
        \item The original GitHub project:  \url{https://github.com/Trinkle23897/THU-Beamer-Theme}, welcome for issues if there are any bugs or requests.
    \end{itemize}
\end{frame}


\section{State of the art}

\subsection{Beamer templates}

\begin{frame}
    \begin{itemize}
        \item Some from \LaTeX{}
        \item Some from Tsinghua University
        \item This template comes from \newline \url{https://www.latexstudio.net/archives/4051.html}
        \item But the very first \href{http://far.tooold.cn/post/latex/beamertsinghua}{\color{purple}{link}} \cite{origin} has already expired.
        \item Some beamers made in 2016-17: \href{https://github.com/Trinkle23897/oi_slides}{\color{erasmusmc!50!white}{click me}}
    \end{itemize}
\end{frame}


\section{Content}

\subsection{Make beamer beautiful}

\begin{frame}{Difference from the original THU theme}
    \begin{itemize}
        \item one-line circle headbars but not multiple
        \item Kaishu font for Chinese (suppressed)
        \item Four years from adaptation so a lot of stuff is forgot
        \item Can refer to this for more usages of this template: \url{https://www.latexstudio.net/archives/4051.html}
        \item Listed below are some usage of this Beamer template, some extracted from \url{https://tuna.moe/event/2018/latex/}
    \end{itemize}
\end{frame}

\subsection{How to make a Beamer better}

\begin{frame}{Why Beamer}
    \begin{itemize}
        \item \LaTeX\ Used broadly in academia and journals
    \end{itemize}
    \begin{table}[h]
        \centering
        \begin{tabular}{c|c}
            Microsoft\textsuperscript{\textregistered}  Word & \LaTeX \\
            \hline
            Lorem ipsum dolor sit amet & Lorem ipsum dolor sit amet \\
            Lorem ipsum dolor sit amet & Lorem ipsum dolor sit amet \\
            Lorem ipsum dolor & Lorem ipsum \\
        \end{tabular}
    \end{table}
\end{frame}

\begin{frame}{Layout example}
    \begin{exampleblock}{plain formula} % add * 
        \begin{equation*}
            J(\theta) = \mathbb{E}_{\pi_\theta}[G_t] = \sum_{s\in\mathcal{S}} d^\pi (s)V^\pi(s)=\sum_{s\in\mathcal{S}} d^\pi(s)\sum_{a\in\mathcal{A}}\pi_\theta(a|s)Q^\pi(s,a)
        \end{equation*}
    \end{exampleblock}
    \begin{exampleblock}{formula with multiple rows\footnote{If there are texts in please use $\backslash$mathrm\{\} or $\backslash$text\{\} to wrap, or it will look like $clip$, much uglier than $\mathrm{clip}$}}
        % using & as separator
        \begin{align}
            Q_\mathrm{target}&=r+\gamma Q^\pi(s^\prime, \pi_\theta(s^\prime)+\epsilon)\\
            \epsilon&\sim\mathrm{clip}(\mathcal{N}(0, \sigma), -c, c)\nonumber
        \end{align}
    \end{exampleblock}
\end{frame}

\begin{frame}
    \begin{exampleblock}{Numbered multiple row formula}
    
        % Taken from Mathmode.tex
        \begin{multline}
            A=\lim_{n\rightarrow\infty}\Delta x\left(a^{2}+\left(a^{2}+2a\Delta x+\left(\Delta x\right)^{2}\right)\right.\label{eq:reset}\\
            +\left(a^{2}+2\cdot2a\Delta x+2^{2}\left(\Delta x\right)^{2}\right)\\
            +\left(a^{2}+2\cdot3a\Delta x+3^{2}\left(\Delta x\right)^{2}\right)\\
            +\ldots\\
            \left.+\left(a^{2}+2\cdot(n-1)a\Delta x+(n-1)^{2}\left(\Delta x\right)^{2}\right)\right)\\
            =\frac{1}{3}\left(b^{3}-a^{3}\right)
        \end{multline}
    \end{exampleblock}
\end{frame}

\begin{frame}{Figure and columns}
    % From THU thesis user guide.
    \begin{minipage}[c]{0.3\linewidth}
        \psset{unit=0.8cm}
        \begin{pspicture}(-1.75,-3)(3.25,4)
            \psline[linewidth=0.25pt](0,0)(0,4)
            \rput[tl]{0}(0.2,2){$\vec e_z$}
            \rput[tr]{0}(-0.9,1.4){$\vec e$}
            \rput[tl]{0}(2.8,-1.1){$\vec C_{ptm{ext}}$}
            \rput[br]{0}(-0.3,2.1){$\theta$}
            \rput{25}(0,0){%
            \psframe[fillstyle=solid,fillcolor=lightgray,linewidth=.8pt](-0.1,-3.2)(0.1,0)}
            \rput{25}(0,0){%
            \psellipse[fillstyle=solid,fillcolor=yellow,linewidth=3pt](0,0)(1.5,0.5)}
            \rput{25}(0,0){%
            \psframe[fillstyle=solid,fillcolor=lightgray,linewidth=.8pt](-0.1,0)(0.1,3.2)}
            \rput{25}(0,0){\psline[linecolor=red,linewidth=1.5pt]{->}(0,0)(0.,2)}
%           \psRotation{0}(0,3.5){$\dot\phi$}
%           \psRotation{25}(-1.2,2.6){$\dot\psi$}
            \psline[linecolor=red,linewidth=1.25pt]{->}(0,0)(0,2)
            \psline[linecolor=red,linewidth=1.25pt]{->}(0,0)(3,-1)
            \psline[linecolor=red,linewidth=1.25pt]{->}(0,0)(2.85,-0.95)
            \psarc{->}{2.1}{90}{112.5}
            \rput[bl](.1,.01){C}
        \end{pspicture}
    \end{minipage}\hspace{1cm}
    \begin{minipage}{0.5\linewidth}
        \medskip
        %\hspace{2cm}
        \begin{figure}[h]
            \centering
            \includegraphics[height=.25\textheight]{pic/erasmusmc_logo.pdf}
        \end{figure}
    \end{minipage}
\end{frame}

\begin{frame}[fragile]{\LaTeX{} Useful commands}
    \begin{exampleblock}{Commands}
        \centering
        \footnotesize
        \begin{tabular}{llll}
            \cmd{chapter} & \cmd{section} & \cmd{subsection} & \cmd{paragraph} \\
            Chapter & Section & Subsection & Para \\\hline
            \cmd{centering} & \cmd{emph} & \cmd{verb} & \cmd{url} \\
            Centered & Emphasized & Verbatim & URL \\\hline
            \cmd{footnote} & \cmd{item} & \cmd{caption} & \cmd{includegraphics} \\
            Footnotes & Enumeration item & Captions & Figures \\\hline
            \cmd{label} & \cmd{cite} & \cmd{ref} \\
            Labels & Citations & Cross-reference\\\hline
        \end{tabular}
    \end{exampleblock}
    \begin{exampleblock}{Environment}
        \centering
        \footnotesize
        \begin{tabular}{lll}
            \env{table} & \env{figure} & \env{equation}\\
            Table & Figure & Formulas \\\hline
            \env{itemize} & \env{enumerate} & \env{description}\\
            Free list & Numbered list & Description \\\hline
        \end{tabular}
    \end{exampleblock}
\end{frame}

\begin{frame}[fragile]{\LaTeX{} Example for environment commands}
    \begin{minipage}{0.5\linewidth}
\begin{lstlisting}[language=TeX]
\begin{itemize}
  \item A \item B
  \item C
  \begin{itemize}
    \item C-1
  \end{itemize}
\end{itemize}
\end{lstlisting}
    \end{minipage}\hspace{1cm}
    \begin{minipage}{0.3\linewidth}
        \begin{itemize}
            \item A
            \item B
            \item C
            \begin{itemize}
                \item C-1
            \end{itemize}
        \end{itemize}
    \end{minipage}
    \medskip
    \pause
    \begin{minipage}{0.5\linewidth}
\begin{lstlisting}[language=TeX]
\begin{enumerate}
  \item BIG \item BIIIIIG
  \item SMall
  \begin{itemize}
    \item[n+e] ??!
  \end{itemize}
\end{enumerate}
\end{lstlisting}
    \end{minipage}\hspace{1cm}
    \begin{minipage}{0.3\linewidth}
        \begin{enumerate}
            \item BIG
            \item BIIIG
            \item SMALl
            \begin{itemize}
                \item[n+e] \#\#\#\#\#
            \end{itemize}
        \end{enumerate}
    \end{minipage}
\end{frame}

\begin{frame}[fragile]{\LaTeX{} Formulae}
    \begin{columns}
        \begin{column}{.55\textwidth}
\begin{lstlisting}[language=TeX]
$V = \frac{4}{3}\pi r^3$

\[
  V = \frac{4}{3}\pi r^3
\]

\begin{equation}
  \label{eq:vsphere}
  V = \frac{4}{3}\pi r^3
\end{equation}
\end{lstlisting}
        \end{column}
        \begin{column}{.4\textwidth}
            $V = \frac{4}{3}\pi r^3$
            \[
                V = \frac{4}{3}\pi r^3
            \]
            \begin{equation}
                \label{eq:vsphere}
                V = \frac{4}{3}\pi r^3
            \end{equation}
        \end{column}
    \end{columns}
    \begin{itemize}
        \item For more see: \href{https://www.erasmusmc.nl}{\color{erasmusmc}{HERE}}
    \end{itemize}
\end{frame}

\begin{frame}[fragile]
    \begin{columns}
        \column{.6\textwidth}
\begin{lstlisting}[language=TeX]
    \begin{table}[htbp]
      \caption{Number and meaning}
      \label{tab:number}
      \centering
      \begin{tabular}{cl}
        \toprule
        Number & Meaning \\
        \midrule
        1 & 4.0 \\
        2 & 3.7 \\
        \bottomrule
      \end{tabular}
    \end{table}
    For number and meaning of 
    formula ~(\ref{eq:vsphere})
    please see 
    Table ~\ref{tab:number}.
\end{lstlisting}
        \column{.4\textwidth}
        \begin{table}[htpb]
            \centering
            \caption{Number and meaning}
            \label{tab:number}
            \begin{tabular}{cl}\toprule
                Number & Meaning \\\midrule
                1 & 4.0\\
                2 & 3.7\\\bottomrule
            \end{tabular}
        \end{table}
        \normalsize For number and meaning of formula ~(\ref{eq:vsphere})
    please see Table ~\ref{tab:number}.
    \end{columns}
\end{frame}

\begin{frame}{Plotting}
    \begin{itemize}
        \item Vector plots eps, ps, pdf
        \begin{itemize}
            \item METAPOST, pstricks, pgf $\ldots$
            \item Xfig, Dia, Visio, Inkscape $\ldots$
            \item Matlab / Excel can be saved as pdf
        \end{itemize}
        \item Scalar plots png, jpg, tiff $\ldots$
        \begin{itemize}
            \item More resolution needed.
            \item Avoided please.
        \end{itemize}
    \end{itemize}
    \begin{figure}[htpb]
        \centering
        \includegraphics[width=0.2\linewidth]{pic/erasmusmc_logo.pdf}
        \caption{This logo is a vector plot}
    \end{figure}
\end{frame}

\section{Time schedule}
\begin{frame}{Time schedule}
    \begin{itemize}
        \item Jan: Lorem ipsum dolor sit amet
        \item Feb: Lorem ipsum dolor sit amet
        \item Mar \& Apr: Lorem ipsum dolor sit amet
        \item May: Lorem ipsum dolor sit amet
    \end{itemize}
\end{frame}

\section{References}

\begin{frame}[allowframebreaks]
    \printbibliography[heading=none]
\end{frame}

\begin{frame}
    \begin{center}
        {\Huge\calligra Thanks!}
    \end{center}
\end{frame}

\end{document}